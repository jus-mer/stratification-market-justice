% Options for packages loaded elsewhere
\PassOptionsToPackage{unicode}{hyperref}
\PassOptionsToPackage{hyphens}{url}
\PassOptionsToPackage{dvipsnames,svgnames,x11names}{xcolor}
%
\documentclass[
  12pt,
  letterpaper,
  DIV=11,
  numbers=noendperiod]{scrartcl}

\usepackage{amsmath,amssymb}
\usepackage{setspace}
\usepackage{iftex}
\ifPDFTeX
  \usepackage[T1]{fontenc}
  \usepackage[utf8]{inputenc}
  \usepackage{textcomp} % provide euro and other symbols
\else % if luatex or xetex
  \usepackage{unicode-math}
  \defaultfontfeatures{Scale=MatchLowercase}
  \defaultfontfeatures[\rmfamily]{Ligatures=TeX,Scale=1}
\fi
\usepackage{lmodern}
\ifPDFTeX\else  
    % xetex/luatex font selection
    \setmainfont[]{Times New Roman}
\fi
% Use upquote if available, for straight quotes in verbatim environments
\IfFileExists{upquote.sty}{\usepackage{upquote}}{}
\IfFileExists{microtype.sty}{% use microtype if available
  \usepackage[]{microtype}
  \UseMicrotypeSet[protrusion]{basicmath} % disable protrusion for tt fonts
}{}
\makeatletter
\@ifundefined{KOMAClassName}{% if non-KOMA class
  \IfFileExists{parskip.sty}{%
    \usepackage{parskip}
  }{% else
    \setlength{\parindent}{0pt}
    \setlength{\parskip}{6pt plus 2pt minus 1pt}}
}{% if KOMA class
  \KOMAoptions{parskip=half}}
\makeatother
\usepackage{xcolor}
\usepackage[margin=2cm]{geometry}
\setlength{\emergencystretch}{3em} % prevent overfull lines
\setcounter{secnumdepth}{5}
% Make \paragraph and \subparagraph free-standing
\makeatletter
\ifx\paragraph\undefined\else
  \let\oldparagraph\paragraph
  \renewcommand{\paragraph}{
    \@ifstar
      \xxxParagraphStar
      \xxxParagraphNoStar
  }
  \newcommand{\xxxParagraphStar}[1]{\oldparagraph*{#1}\mbox{}}
  \newcommand{\xxxParagraphNoStar}[1]{\oldparagraph{#1}\mbox{}}
\fi
\ifx\subparagraph\undefined\else
  \let\oldsubparagraph\subparagraph
  \renewcommand{\subparagraph}{
    \@ifstar
      \xxxSubParagraphStar
      \xxxSubParagraphNoStar
  }
  \newcommand{\xxxSubParagraphStar}[1]{\oldsubparagraph*{#1}\mbox{}}
  \newcommand{\xxxSubParagraphNoStar}[1]{\oldsubparagraph{#1}\mbox{}}
\fi
\makeatother


\providecommand{\tightlist}{%
  \setlength{\itemsep}{0pt}\setlength{\parskip}{0pt}}\usepackage{longtable,booktabs,array}
\usepackage{calc} % for calculating minipage widths
% Correct order of tables after \paragraph or \subparagraph
\usepackage{etoolbox}
\makeatletter
\patchcmd\longtable{\par}{\if@noskipsec\mbox{}\fi\par}{}{}
\makeatother
% Allow footnotes in longtable head/foot
\IfFileExists{footnotehyper.sty}{\usepackage{footnotehyper}}{\usepackage{footnote}}
\makesavenoteenv{longtable}
\usepackage{graphicx}
\makeatletter
\def\maxwidth{\ifdim\Gin@nat@width>\linewidth\linewidth\else\Gin@nat@width\fi}
\def\maxheight{\ifdim\Gin@nat@height>\textheight\textheight\else\Gin@nat@height\fi}
\makeatother
% Scale images if necessary, so that they will not overflow the page
% margins by default, and it is still possible to overwrite the defaults
% using explicit options in \includegraphics[width, height, ...]{}
\setkeys{Gin}{width=\maxwidth,height=\maxheight,keepaspectratio}
% Set default figure placement to htbp
\makeatletter
\def\fps@figure{htbp}
\makeatother
% definitions for citeproc citations
\NewDocumentCommand\citeproctext{}{}
\NewDocumentCommand\citeproc{mm}{%
  \begingroup\def\citeproctext{#2}\cite{#1}\endgroup}
\makeatletter
 % allow citations to break across lines
 \let\@cite@ofmt\@firstofone
 % avoid brackets around text for \cite:
 \def\@biblabel#1{}
 \def\@cite#1#2{{#1\if@tempswa , #2\fi}}
\makeatother
\newlength{\cslhangindent}
\setlength{\cslhangindent}{1.5em}
\newlength{\csllabelwidth}
\setlength{\csllabelwidth}{3em}
\newenvironment{CSLReferences}[2] % #1 hanging-indent, #2 entry-spacing
 {\begin{list}{}{%
  \setlength{\itemindent}{0pt}
  \setlength{\leftmargin}{0pt}
  \setlength{\parsep}{0pt}
  % turn on hanging indent if param 1 is 1
  \ifodd #1
   \setlength{\leftmargin}{\cslhangindent}
   \setlength{\itemindent}{-1\cslhangindent}
  \fi
  % set entry spacing
  \setlength{\itemsep}{#2\baselineskip}}}
 {\end{list}}
\usepackage{calc}
\newcommand{\CSLBlock}[1]{\hfill\break\parbox[t]{\linewidth}{\strut\ignorespaces#1\strut}}
\newcommand{\CSLLeftMargin}[1]{\parbox[t]{\csllabelwidth}{\strut#1\strut}}
\newcommand{\CSLRightInline}[1]{\parbox[t]{\linewidth - \csllabelwidth}{\strut#1\strut}}
\newcommand{\CSLIndent}[1]{\hspace{\cslhangindent}#1}

\usepackage[noblocks]{authblk}
\renewcommand*{\Authsep}{, }
\renewcommand*{\Authand}{, }
\renewcommand*{\Authands}{, }
\renewcommand\Affilfont{\small}
\KOMAoption{captions}{tableheading}
\makeatletter
\@ifpackageloaded{caption}{}{\usepackage{caption}}
\AtBeginDocument{%
\ifdefined\contentsname
  \renewcommand*\contentsname{Table of contents}
\else
  \newcommand\contentsname{Table of contents}
\fi
\ifdefined\listfigurename
  \renewcommand*\listfigurename{List of Figures}
\else
  \newcommand\listfigurename{List of Figures}
\fi
\ifdefined\listtablename
  \renewcommand*\listtablename{List of Tables}
\else
  \newcommand\listtablename{List of Tables}
\fi
\ifdefined\figurename
  \renewcommand*\figurename{Figure}
\else
  \newcommand\figurename{Figure}
\fi
\ifdefined\tablename
  \renewcommand*\tablename{Table}
\else
  \newcommand\tablename{Table}
\fi
}
\@ifpackageloaded{float}{}{\usepackage{float}}
\floatstyle{ruled}
\@ifundefined{c@chapter}{\newfloat{codelisting}{h}{lop}}{\newfloat{codelisting}{h}{lop}[chapter]}
\floatname{codelisting}{Listing}
\newcommand*\listoflistings{\listof{codelisting}{List of Listings}}
\makeatother
\makeatletter
\makeatother
\makeatletter
\@ifpackageloaded{caption}{}{\usepackage{caption}}
\@ifpackageloaded{subcaption}{}{\usepackage{subcaption}}
\makeatother

\ifLuaTeX
  \usepackage{selnolig}  % disable illegal ligatures
\fi
\usepackage{bookmark}

\IfFileExists{xurl.sty}{\usepackage{xurl}}{} % add URL line breaks if available
\urlstyle{same} % disable monospaced font for URLs
\hypersetup{
  pdftitle={El rol de la meritocracia, la percepción de desigualdad y la clase social en la justificación de pensiones desiguales en Chile (2016-2023)},
  pdfauthor={Kevin Carrasco; Juan Carlos Castillo; Andreas Laffert; René Canales; Tomás Urzúa},
  colorlinks=true,
  linkcolor={blue},
  filecolor={Maroon},
  citecolor={Blue},
  urlcolor={Blue},
  pdfcreator={LaTeX via pandoc}}


\title{El rol de la meritocracia, la percepción de desigualdad y la
clase social en la justificación de pensiones desiguales en Chile
(2016-2023)}
\author{Kevin Carrasco \and Juan Carlos Castillo \and Andreas
Laffert \and René Canales \and Tomás Urzúa}
\date{}

\begin{document}
\maketitle
\begin{abstract}
\newline

\textbf{Keywords}: Pensions, Social class, meritocracy, market justice,
Chile
\end{abstract}


\setstretch{1.15}
El sistema de pensiones chileno ha sido ampliamente analizado como un
caso extremo de privatización de la protección social en América Latina
(\citeproc{ref-arenasdemesa_sistemas_2019}{Arenas de Mesa, 2019};
\citeproc{ref-castiglioni_politics_2013}{Castiglioni, 2013}). Desde su
instauración en 1981, el diseño basado en la capitalización individual
ha transferido los riesgos a los trabajadores, profundizando
trayectorias laborales fragmentadas y precarizadas. Esto ha derivado en
tasas de reemplazo insuficientes para vastos sectores de la población,
manteniendo una promesa de bienestar que se sostiene sobre principios de
mercado (\citeproc{ref-castiglioni_explaining_2018}{Castiglioni, 2018}).
Así, la idea de que mayores ingresos en la etapa laboral justifican
mejores pensiones refleja cómo la desigualdad económica se legitima
socialmente y se prolonga a través de la estratificación de ingresos y
estatus durante todo el ciclo de vida.

La justicia de mercado se ha conceptualizado como un principio normativo
que legitima la distribución desigual de bienes y servicios al basarse
en la idea de que los resultados económicos reflejan méritos y
responsabilidades individuales
(\citeproc{ref-barry_theories_1989}{Barry, 1989};
\citeproc{ref-lane_market_1986}{Lane, 1986}). Desde esta perspectiva,
las desigualdades no solo se consideran inevitables, sino justas,
siempre que surjan de reglas transparentes de competencia y capacidad de
pago. Esta visión contrasta con enfoques que entienden derechos sociales
como garantías universales de bienestar. La evidencia comparada muestra
que, en países con una mayor mercantilización de servicios como la
salud, la justicia de mercado se convierte en un marco cultural que
forma expectativas y naturaliza brechas de acceso y calidad
(\citeproc{ref-immergut_it_2020}{Immergut \& Schneider, 2020}).

La forma en que se legitima la justicia de mercado tiende a estar
estructurada por la posición que ocupan las personas en la jerarquía
social. Distintos estudios muestran que quienes se ubican en los
estratos de mayores ingresos tienden a justificar desigualdades
distributivas, ya que interpretan sus ventajas como legítimas y
consistentes con trayectorias de éxito
(\citeproc{ref-koos_moral_2019}{Koos \& Sachweh, 2019}). La literatura
comparada indica que, en contextos de alta desigualdad de ingresos y
baja movilidad social percibida, la estratificación de clase refuerza la
aceptación de principios de mercado para explicar por qué mayores
ingresos se asocian con mejores beneficios y protección social
(\citeproc{ref-trump_income_2018}{Trump, 2018}). Esta interacción entre
posición social y justificación de la desigualdad muestra cómo la
estructura de clase moldea creencias normativas sobre lo que se
considera justo.

Además de la posición de clase, las creencias meritocráticas refuerzan
la aceptación de la justicia de mercado, pues la meritocracia se
entiende como un principio normativo que justifica la asignación de
recompensas y oportunidades en función del esfuerzo y el talento
individuales (\citeproc{ref-young_rise_1962}{Young, 1962}). Mijs
(\citeproc{ref-mijs_paradox_2021}{2021}) subraya que esta promesa
resulta paradójica en contextos de alta desigualdad estructural, ya que
tiende a invisibilizar barreras sistémicas. La literatura muestra que
estas creencias se transmiten y reproducen a través de distintos
espacios de socialización ---la educación, el trabajo y los discursos
políticos--- reforzando la disposición a legitimar resultados desiguales
como reflejo de mérito
(\citeproc{ref-castillo_socialization_2024}{Castillo et al., 2024};
\citeproc{ref-heuer_legitimizing_2020}{Heuer et al., 2020};
\citeproc{ref-reynolds_perceptions_2014}{Reynolds \& Xian, 2014}).

La percepción de desigualdad modula la legitimidad de resultados
desiguales: cuando las brechas reales se subestiman, la justicia de
mercado y las creencias meritocráticas se sostienen con mayor facilidad
(\citeproc{ref-gimpelson_misperceiving_2018}{Gimpelson \& Treisman,
2018}). En cambio, percepciones más ajustadas pueden tensionar estos
principios y debilitar la justificación de resultados desiguales, como
la aceptación de pensiones diferenciadas. De este modo, la percepción de
desigualdad actúa como un factor clave que puede reforzar o erosionar la
coherencia entre estratificación social, meritocracia y justicia de
mercado (\citeproc{ref-janmaat_subjective_2013}{Janmaat, 2013}).

Utilizando datos del Estudio Longitudinal Social de Chile (2016-2023),
esta investigación pretende analizar cómo aceptación de mejores
pensiones para personas de mayores ingresos se articula mediante la
interacción de la posición de clase, las creencias meritocráticas y la
percepción de desigualdad. Esta perspectiva contribuye a comprender cómo
se legitima la reproducción de desigualdades estructurales en la vejez
dentro de un régimen de justicia de mercado, en el que los intereses
privados siguen limitando la adopción de políticas más redistributivas.
Así, los resultados aportan evidencia para debatir la sostenibilidad de
modelos previsionales altamente mercantilizados y sus tensiones con la
cohesión social.

Referencias

\phantomsection\label{refs}
\begin{CSLReferences}{1}{0}
\bibitem[\citeproctext]{ref-arenasdemesa_sistemas_2019}
Arenas de Mesa, A. (2019). \emph{Los sistemas de pensiones en la
encrucijada: Desaf{í}os para la sostenibilidad en {Am{é}rica Latina}}
(Libros de la CEPAL N{\(^\circ\)} 159 (LC/PUB.2019/19-P)). Santiago de
Chile: CEPAL.

\bibitem[\citeproctext]{ref-barry_theories_1989}
Barry, B. (1989). \emph{Theories of justice}. Berkeley: Univ. of
California Pr.

\bibitem[\citeproctext]{ref-castiglioni_politics_2013}
Castiglioni, R. (Ed.). (2013). \emph{The politics of social policy
change in {Chile} and {Uruguay}: Retrenchment versus maintenance,
1973-1998}. Milton Park, Abingdon, Oxon New York, NY: Routledge, Taylor
\& Francis Group.

\bibitem[\citeproctext]{ref-castiglioni_explaining_2018}
Castiglioni, R. (2018). Explaining {Uneven Social Policy Expansion} in
{Democratic Chile}. \emph{Latin American Politics and Society},
\emph{60}(3), 54--76. \url{https://doi.org/10.1017/lap.2018.24}

\bibitem[\citeproctext]{ref-castillo_socialization_2024}
Castillo, J. C., Salgado, M., Carrasco, K., \& Laffert, A. (2024). The
{Socialization} of {Meritocracy} and {Market Justice Preferences} at
{School}. \emph{Societies}, \emph{14}(11), 214.
\url{https://doi.org/10.3390/soc14110214}

\bibitem[\citeproctext]{ref-gimpelson_misperceiving_2018}
Gimpelson, V., \& Treisman, D. (2018). Misperceiving inequality.
\emph{Economics \& Politics}, \emph{30}(1), 27--54.
\url{https://doi.org/10.1111/ecpo.12103}

\bibitem[\citeproctext]{ref-heuer_legitimizing_2020}
Heuer, J.-O., Lux, T., Mau, S., \& Zimmermann, K. (2020). Legitimizing
{Inequality}: {The Moral Repertoires} of {Meritocracy} in {Four
Countries}. \emph{Comparative Sociology}, \emph{19}(4-5), 542--584.
\url{https://doi.org/10.1163/15691330-bja10017}

\bibitem[\citeproctext]{ref-immergut_it_2020}
Immergut, E. M., \& Schneider, S. M. (2020). Is it unfair for the
affluent to be able to purchase {``better''} healthcare? {Existential}
standards and institutional norms in healthcare attitudes across 28
countries. \emph{Social Science \& Medicine}, \emph{267}, 113146.
\url{https://doi.org/10.1016/j.socscimed.2020.113146}

\bibitem[\citeproctext]{ref-janmaat_subjective_2013}
Janmaat, J. G. (2013). Subjective inequality: {A} review of
international comparative studies on people's views about inequality.
\emph{Archives Europeennes de Sociologie}, \emph{54}(3), 357--389.
\url{https://doi.org/10.1017/S0003975613000209}

\bibitem[\citeproctext]{ref-koos_moral_2019}
Koos, S., \& Sachweh, P. (2019). The moral economies of market
societies: Popular attitudes towards market competition, redistribution
and reciprocity in comparative perspective. \emph{Socio-Economic
Review}, \emph{17}(4), 793--821.
\url{https://doi.org/10.1093/ser/mwx045}

\bibitem[\citeproctext]{ref-lane_market_1986}
Lane, R. E. (1986). Market {Justice}, {Political Justice}.
\emph{American Political Science Review}, \emph{80}(2), 383--402.
\url{https://doi.org/10.2307/1958264}

\bibitem[\citeproctext]{ref-mijs_paradox_2021}
Mijs, J. (2021). The paradox of inequality: Income inequality and belief
in meritocracy go hand in hand. \emph{Socio-Economic Review},
\emph{19}(1), 7--35. \url{https://doi.org/10.1093/ser/mwy051}

\bibitem[\citeproctext]{ref-reynolds_perceptions_2014}
Reynolds, J., \& Xian, H. (2014). Perceptions of meritocracy in the land
of opportunity. \emph{Research in Social Stratification and Mobility},
\emph{36}, 121--137. \url{https://doi.org/10.1016/j.rssm.2014.03.001}

\bibitem[\citeproctext]{ref-trump_income_2018}
Trump, K.-S. (2018). Income {Inequality Influences Perceptions} of
{Legitimate Income Differences}. \emph{British Journal of Political
Science}, \emph{48}(4), 929--952.
\url{https://doi.org/10.1017/S0007123416000326}

\bibitem[\citeproctext]{ref-young_rise_1962}
Young, M. (1962). \emph{The rise of the meritocracy}. Baltimore: Penguin
Books.

\end{CSLReferences}




\end{document}
